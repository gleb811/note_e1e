\chapter{Physics motivation}
\label{motivation}

In this analysis note new set of differential and fully integrated cross sections for the exclusive reaction
 $ep \to e'p' \pi^+ \pi^-$ from data of the e1e run collected with the CLAS detector in the kinematic area of 
1.3~GeV~$< W <$~1.825~GeV and 0.4~GeV$^{2}$~$< Q^{2} <$~1.0~GeV$^{2}$ is presented. In each bin of $W$ and $Q^2$ nine single-differential cross sections are obtained. They consist of: a) three distributions over invariant masses of the final hadron pairs; b) three CM-angular distributions over polar angles $\theta$ of the final $\pi^+$, $\pi^-$, and $p'$, and c) three CM-angular distributions over the angles $\alpha$ between two planes. One plain is defined by the three-momenta of all final hadrons. Another plane is defined by the three-momenta of virtual photon and one of the final hadrons for the three different choices of this final hadron.
More detailed information about kinematical variables is in Sect.~\ref{kin_var}.
These data were obtained for the first time in the kinematic area $Q^{2} < 0.6$~GeV$^{2}$ and $W > 1.55$~GeV. At 0.6~GeV$^{2}$~$< Q^{2} <$~1.0~GeV$^{2}$  similar $\pi^+\pi^-p$ single-differential cross sections have already been measured with CLAS \cite{Ripani:2002ss}. However, in this data set these cross sections are obtained  in $Q^2$-bins of bin sizes, which are roughly a factor of six smaller than those achieved in~\cite{Ripani:2002ss}. 


The studies of exclusive $\pi^+ \pi^- $ electroproduction off protons represent an important avenue in the investigation of the
$N^*$ spectrum and structure via analyses of experimental data on exclusive meson electroproduction with CLAS. The CLAS detector has provided the 
dominant portion of all data on meson electroproduction in the resonance excitation region. The studies of 
transition helicity amplitudes from the proton ground state to its excited states represent a key aspect of
the $N^*$ program with CLAS~\cite{Aznauryan:2011qj,Aznauryan:2011ub}. Data on meson electroproduction off nucleons in the $N^*$ region obtained with CLAS open an opportunity to 
determine the $Q^2$ evolution of the $\gamma_vN \rightarrow N^*$ electrocouplings both in comparative and combined analyses of various meson electroproduction channels. The electroexcitation amplitudes for the low-lying resonances $\Delta(1232)3/2^+$, $N(1440)1/2^+$, $N(1520)3/2^-$,
and $N(1535)1/2^-$ have been determined over a wide range of $Q^2$ in a comprehensive analysis of JLab-CLAS data 
on differential cross sections, longitudinally polarized beam asymmetries, and 
beam-target asymmetries for single pion electroproduction off protons~\cite{Aznauryan:2009mx}. Recently $\gamma_v N \rightarrow N^*$ electrocouplings of several higher-lying nucleon 
resonances: $N(1675)5/2^-$, $N(1680)5/2^+$, and $N(1710)1/2^+$ have become available for the first time for 
1.5~GeV$^2$ $< Q^2 <$ 4.5~GeV$^2$ from the analysis of exclusive $\pi^+$ electroproduction off the proton~\cite{Park:2014yea}. 
Electrocouplings for the $N(1440)1/2^+$ and $N(1520)3/2^-$ resonances for $Q^2 < 0.6$~GeV$^2$ have been 
determined from the data~\cite{Fedotov:2008aa} on exclusive $\pi^+\pi^-$ electroproduction off the proton~\cite{Mokeev:2012vsa}. The recent analysis~\cite{Mokeev:2015lda}
of the CLAS data on $\pi^+ \pi^- $ electroproduction off protons \cite{Ripani:2002ss} provided the results on electrocouplings of these states in a wider $Q^2$-range up to 1.5 GeV$^2$. Furthermore, electrocouplings of the $\Delta(1620)3/2^-$ resonance that decays preferentially to the $N\pi\pi$ final states have become available from this analysis for the first time.
Consistent results for the $\gamma_vp \rightarrow N^*$ electrocouplings of the $N(1440)1/2^+$ and  $N(1520)3/2^-$ resonances, that have been determined in independent analyses of the dominant meson electroproduction channels $N\pi$ and $\pi^+\pi^-p$ with completely different non-resonant contributions,  demonstrated the reliable extraction of these fundamental quantities. This success also supports the capability of the reaction models, that have been developed for the extraction of the resonance parameters from the analyses of data on single- \cite{Aznauryan:2009mx} and double-pion \cite{Mokeev:2008iw} electroproduction off protons, to provide reliable information on the $N^*$ parameters from independent studies of either of these major exclusive channels.





The CLAS results on the $\gamma_vp \rightarrow N^*$ electrocouplings~\cite{Aznauryan:2009mx,Aznauryan:2011qj,Mokeev:2012vsa,Mokeev:2013kka,Park:2014yea,Mokeev:2015lda} have had a stimulating 
impact on the theory of the excited nucleon state structure, in particular, on QCD-based approaches. The light cone sum rule (LCSR) approach~\cite{Anikin:2015ita,Braun:2014wpa} for the first time provided access to the quark 
distribution amplitudes (DAs) inside the $N(1535)1/2^-$ resonance from analysis of the CLAS results on the
$\gamma_vp \rightarrow N^*$ electrocouplings of this state~\cite{Aznauryan:2009mx}. Confronting the quark DAs of excited nucleon states 
determined from the experimental results on the $\gamma_vp \rightarrow N^*$ electrocouplings to the LQCD expectations, makes 
it possible to explore the emergence of the resonance structure starting from the QCD Lagrangian. The moments 
of the $N(1535)1/2^-$ quark DAs derived from the CLAS data are consistent with the LQCD expectations~\cite{Braun:2014wpa}. The Dyson-Schwinger Equations of QCD (DSEQCD) provide a conceptually different avenue for relating the 
$\gamma_vp  \rightarrow N^*$ electrocouplings to the fundamental QCD Lagrangian~\cite{Segovia:2014aza,Binosi:2014aea,Segovia:2015hra}. The DSEQCD 
approach allows to evaluate the contribution of the three bound dressed quarks, the so-called quark 
core, to the structure of excited nucleon states starting from the QCD Lagrangian. A successful description of 
the nucleon elastic form factors and the CLAS results on the $N \to \Delta$, $N \to N(1440)1/2^+$ electromagnetic transition 
 form factors~\cite{Aznauryan:2009mx,Mokeev:2012vsa,Aznauryan:2011qj,Mokeev:2013kka} at photon virtualities $Q^2 > 2.0$~GeV$^2$ has been 
recently achieved within the DSEQCD framework~\cite{Cloet:2013gva,Segovia:2014aza,Segovia:2015hra}. This successful 
description of the form factors that correspond to distinctively different 
structures achieved with the same dressed quark mass function strongly underlines:
\begin{itemize}
\item the relevance of dynamical dressed quarks with the properties predicted by the DSEQCD approach~\cite{Cloet:2013jya} 
as constituents of the quark cores for the structure of the ground and excited nucleon states;
\item the capability of the DSEQCD approach~\cite{Segovia:2014aza,Segovia:2015hra} to map out the dressed quark mass function from 
the experimental results on the $Q^2$ evolution of the nucleon elastic and $p \to N^*$ transition form 
factors ($\gamma_vp \rightarrow N^*$ electrocouplings) from a combined analysis.
\end{itemize}
 
Physics analyses of the CLAS results~\cite{Aznauryan:2009mx,Mokeev:2012vsa,Mokeev:2013kka} on the $\gamma_vp \rightarrow N^*$ electrocouplings revealed the 
structure of excited nucleon states at photon virtualities $Q^2 < 5.0$~GeV$^2$ as a complex interplay between 
meson-baryon and quark degrees of freedom. The relative contributions from the meson-baryon cloud and the quark 
core are strongly dependent on the quantum numbers of the excited nucleons. Analyses of the $A_{1/2}$ 
electrocouplings of the $N(1520)3/2^-$ resonance~\cite{JuliaDiaz:2007fa,Santopinto:2012nq} demonstrated that this amplitude is dominated 
by quark core contributions in the entire range of $Q^2 < 5.0$~GeV$^2$ measured by CLAS. However, the recent 
analysis~\cite{Aznauryan:2014xea} of the first CLAS results~\cite{Park:2014yea} on the $N(1675)5/2^-$ $\gamma_vp \rightarrow N^*$ electrocouplings 
suggested a dominance of the meson-baryon cloud. Pronounce differences in the structure of the $N^*$ states of different quantum numbers demonstrate different manifestations of the non-perturbative strong interaction in generation of excited nucleons as the bound systems of an infinite amount of quarks and gluons. The studies of $\gamma_vp \rightarrow N^*$ electrocouplings for all prominent nucleon resonances offer unique information on many facets of the non-perturbative strong interaction and motivate the extension of the studies of $\gamma_vp \rightarrow N^*$ electrocouplings over full spectrum of excited nucleons.  


Currently the results on resonance electrocouplings in mass range above 1.6 GeV are rather limited. The recent studies of
single pion electroproduction~\cite{Park:2014yea} delivered the results on electrocouplings of only those high mass states, which have a substantial branching fraction for decays to the $N\pi$ final states. Several high-lying nucleon excitations, as $\Delta(1620)1/2^-$,
$\Delta(1700)3/2^-$, and $N(1720)3/2^+$, decay preferentially to $N\pi\pi$ final states making the channel of $\pi^+\pi^-$ electroproduction off the proton the major source of information on electrocouplings of these states. In the future these electrocouplings can be checked in independent analyses of $KY$ exclusive electroproduction channels~\cite{Carman:2016hlp}. The data on electrocouplings of high-lying resonances, which decay both to the $N\pi$ and the $N\pi\pi$ final states, in  $\pi^+\pi^-$ electroproduction channel will make it possible to test the consistency of these results with those from independent analyses of single pion electroproduction, offering a sensitive check of the reliability of the resonance parameter extraction.




The experimental data on $\pi^+\pi^-$ electroproduction off protons presented in this analysis note will be analyzed within the framework of the meson-baryon reaction model JM~\cite{Mokeev:2008iw,Mokeev:2012vsa,Mokeev:2015lda}, which is currently the only available approach worldwide for the extraction of resonance electrocouplings from this exclusive channel. The aforementioned approach already provided reliable results on electrocouplings of all resonances in mass range below 1.65~GeV with sizable decays to the $N\pi\pi$ final states. The analysis of experimental data presented in this note eventually will allow us to:
\begin{itemize}
\item determine the evolution of the electrocouplings of most nucleon resonances in mass range up to 1.825~GeV with photon virtualities $Q^2$ up to 1.0~GeV$^2$  with bin sizes in $Q^2$ much smaller than previously achieved in any experiments. For high-lying resonances that decay preferentially to the $N\pi\pi$ final states this information will be obtained for the first time;
\item explore electrocouplings of all orbital excitations (L=1) of the [70,1$^-$] spin-flavor $SU(6)$-supermultiplet in a combined analysis of the results from both $N\pi$ and $N\pi\pi$ channels.
\end{itemize} 

Studies of the combined CLAS preliminary results on $\pi^+\pi^-$ photo- and electroproduction \cite{Ripani:2002ss} within the framework of the JM meson-baryon reaction model have provided further convincing evidences for the existence of the new baryon state
$N'(1720)3/2^+$ \cite{Mokeev:2015moa}. So far, it is the only candidate state, for which information on the internal structure has become available from results on $\gamma_vp \rightarrow N^*$ electrocouplings at $Q^2$~$<$ 1.5~GeV$^2$. However, right now, only four data points for this new state's electrocouplings are available. From the data of this analysis note, the $N'(1720)3/2^+$ electrocoupling values will become available at the set of additional $Q^2$ bins of bin size at least a factor of  six smaller than available from the old CLAS experiment~\cite{Ripani:2002ss}. The expected results will for the first time offer a deep insight to the structure of the new baryon state  $N'(1720)3/2^+$.

The expected results will extend considerably the available information on the interplay between meson-baryon cloud and quark core contributions to the structure of excited nucleon states in particular in mass range from 1.6 to 1.8~GeV.

