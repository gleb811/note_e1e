\chapter{Physics motivation}
%\addcontentsline{toc}{chapter}{Physics motivation}

The main goal of this analysis is to determining nine 1-fold differential 2$\pi$ electroproduction cross sections in still uncovered kinematic area of $1.35 < W < 1.8$~GeV and $0.3 < Q^{2} < 1.0$~GeV$^{2}$. From e1e data set for the first time 2$\pi$ cross sections at $Q^{2} < 0.6$~GeV$^{2}$ and $W > 1.55$~GeV was obtained. In the area of $0.6 < Q^{2} < 1.0$~GeV$^{2}$, that overlaps with available CLAS 2$\pi$ data, cross sections was obtained  with more than factor two smaller binning over $Q^{2}$ in comparison with what was achieved so far. Eventually this data allows to determine $Q^{2}$ evolution of electrocouplings for almost all excited proton states with masses less than 1.8~GeV at $Q^{2}$ from 0.3 to 1.0 GeV2. For the first time the information on electrocouplings of almost all N*'s in second and third resonance regions will become available at photon virtualities, that correspond to the contributions from both meson-baryon dressing and quark degrees of freedom. Therefore, this information is vital in order to explore the transition from meson-baryon to quark degrees of freedom in excited proton states.

Victor please add more.
